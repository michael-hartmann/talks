\documentclass[
%	pdftex,%              PDFTex verwenden
	a4paper,%             A4 Papier
	twoside,%             Zweiseitig/Einseitigtotocnumbered,%    Literaturverzeichnis nummeriert einfügen
%	idxtotoc,%            Index ins Verzeichnis einfügen
	index=totoc,
%	halfparskip,%         Europäischer Satz mit abstand zwischen Absätzen
	parskip=half,
 	chapterprefix,%       Kapitel anschreiben als Kapitel
	headsepline,%         Linie nach Kopfzeile
%	footsepline,%         Linie vor Fusszeile
	13pt,%                Größere Schrift, besser lesbar am bildschrim
	BCOR5mm,%							5mm Abstand Rand
	fleqn,%								linksbündig abgesetzte Formeln
  openany,%              keine leeren seiten zwischen kapiteln
  ngerman
]{scrbook}

\usepackage[ngerman]{babel}
\usepackage{setspace}
\usepackage{a4wide}					 				%Bessere Ausnutzung der DinA4 größe ohne Randbemerkungen, auch a4 möglich
\usepackage[T1]{fontenc}						%Europäischer Schriftsatz
\usepackage[utf8]{inputenc}
\usepackage{graphicx}								%Einbindung von Graphiken
\usepackage{amssymb,amsmath}				%Mathe und Schriftsatz
%\usepackage{amsfonts}
\usepackage{prettyref}
\usepackage{xspace}
\usepackage{colortbl}
\usepackage[format=plain,margin=1.5cm,small]{caption} %Legenden: mergin versetzt beitseitig nach innen, format=plain verhindert einrücken (einfacher blocksatz)

%Graphikerweiterungen für eps-Graphiken
\usepackage{subfigure}
%\usepackage{thumbpdf}
\usepackage{epsfig}
\usepackage{rotating}

\setlength{\parindent}{0.0cm}				%Abstand bei Absatzeinrücken (falls unerwünscht, dann \noident)

%\usepackage[numbers,sort&compress]{natbib}           % Einfache Anpassung der Bibliographie und Zitierstile
%\usepackage{eurosym}             % Euro-Symbol (offiziell)
%
% Type 1 Fonts für bessere darstellung in PDF verwenden.
%
\usepackage{courier}            % Courier als \ttdefault
%\usepackage{}    				
% skalierte Helvetica als \sfdefault [scaled=1] (avant,helvet(mit scaled=...), bookman, utopia, charter, pifont, newcent)


\setkomafont{sectioning}{\normalfont\bfseries}
\setkomafont{captionlabel}{\normalfont\bfseries}
%\setkomafont{pagehead}{\normalfont\itshape}
\setkomafont{descriptionlabel}{\normalfont\bfseries}

% Caption-Stil
% \setlength{\captionindent}{3cm}
% \renewcommand{\captionlabelfont}{\bfseries \sffamily}

% Tabellen:

% gr\"{o}{\ss}ere Zeilenh\"{o}he
%\setlength{\extrarowheight}{0.2cm}

% Neue Spaltenstile, f\"{u}r Dezimalzahlen
\newcolumntype{1}{D{.}{.}{1.13}}
% und grau hinterlegte Tabellenzellen
\newcolumntype{G}{>{\columncolor[gray]{0.8}}c}

% Abk\"{u}rzung f\"{u}r graue Tabellenzelle
\newcommand{\GS}[1]{\multicolumn{1}{G}{#1}}

% Farbe zwischen Doppellinien aus \hhline:
\doublerulesepcolor{white}


\usepackage{txfonts}
\usepackage[scaled=.92]{helvet}
\usepackage{courier}


% Tabelleneinbindung
\usepackage{array}
\usepackage{float}

	%1,5 Zeilenabstand
  \renewcommand{\baselinestretch}{1.1}

	%Seite einrichten
	%\textwidth15cm
	%\topmargin0pt
	%\oddsidemargin0.86cm
	%\evensidemargin0.06cm
	%\headheight0.7cm
	%\headsep1cm
	%\textheight22.7cm
	%\footskip0.7cm
	
	%Kopf und Fusszeile einrichten
%	\markright{\thesection}
%	\pagestyle{headings}
%\setlength{\oddsidemargin}{0.46cm}
%\setlength{\textwidth}{16cm}


% Mehr Platz für Bilder auf den Seiten:
\renewcommand{\topfraction}{.95}
\renewcommand{\bottomfraction}{.95}
\renewcommand{\textfraction}{.05}

% \setlength{\intextsep}{5ex plus 1ex minus 1ex}

% Erstellung von Index
\usepackage{makeidx}
\makeindex
\definecolor{ForestGreen}{rgb}{0, 0.545, 0} %definiert dunkleres grün mit name in rgb (werte zwischen 0 und 1)
\usepackage{hyphenat} %Silbentrennung verhindern mit: \nohyphens{***}

\usepackage{hyperref}  % unbedingt als letztes Paket laden

\title{Physik}
\author{Pollak Christian}
\date{\today}

\newcommand{\mb}{\mu_\mathrm{B}} 

\begin{document}
	

\subsection{Übersicht}
ESR, NMR:
\begin{itemize}
\item resonante Absorption von Mikrowellenstrahlung durch paramagn. Ionen, Moleküle$\dots$ in einem statischen Magnetfeld
\item Aufspaltung der Energieniveaus durch statisches Magnetfeld
\item Anregung von Übergängen zwischen Energieniveaus durch oszillierendes Magnetfeld
\item ESR: magnet. Dipole nötig, Oszillationsfrequenz im Mikrowellenbereich
\end{itemize}

Bloch-Gleichungen:
\begin{itemize}
\item Bewegungsgleichungen für Magnetisierung $\vec M$
\item Änderung der Magnetisierung in Abhängigkeit äußerer Felder
\item phänomenologische, einfachst mögliche Beschreibung
\end{itemize}


\subsection{Übersicht}

\begin{itemize}
\item für ESR magnet. Dipole notwendig: Spin, Drehmoment
\item magnet. Moment: Maß für die Stärke eines Dipols
\item magnet. Moment eines Atoms: LS-Kopplung: 
\begin{equation}
\vec \mu = - \mb \left(\vec L + g \vec S\right) = -\gamma \hbar \vec J
\end{equation}
\item negatives Vorzeichen: Elektronen
\item g: Lande-Faktor, gyromagnetisches Verhältnis
\item klassische Rechnung $\mb$
\begin{align}
\hbar L &= m v r \\
\mu &= I A = \frac{\mathrm{e}}{2\pi r/v} \pi r^2 = \frac{evr}{2} = \frac{e \hbar}{2m} L = \mb L
\end{align}
\item S, L, J aus Hundschen Regeln
\item LS-Kopplung (Spin-Bahn--Kopplung) -- aus $\vec L$ und $\vec S$: $\vec J$
\item Hamilton-Operator: $\mathcal{H} = \frac{1}{2m} \left(\vec p - q\vec A \right)^2$, $\vec B = \nabla \times \vec A$, $\vec A = -\frac{1}{2} \vec r \times \vec B$
  \item Hamilton-Operator 
  \begin{equation}
  \mathcal{H} = -\vec\mu \cdot \vec H = g \mb \vec H \cdot \vec J = g\mb H J_z
  \end{equation}
\end{itemize}

\subsection{Zeeman-Aufspaltung}

\begin{itemize}
\item $J+1$-fache Aufspaltung
\item Aufhebung der Entartung
\item äquidistante Aufspaltungen
\item nur gültig bei schwachen Feldern
\end{itemize}

\subsection{Bewegungsgleichungen}

\begin{itemize}
\item Integration d$\tau$ ueber Raum und Spins
\item Erwartungswert eines Dipols
\item Heisenbersche Bewegungsgleichungen, Ehrenfest-Theorem
\item $\mu$ ist nicht zeitabhängig $\Rightarrow$ Einsetzen
    \begin{align}
    \frac{\mathrm{d}}{\mathrm{d}t} \langle \vec \mu \rangle &= \frac{\mathrm{i}}{\hbar} \langle \left[\mathcal{H}, \vec \mu\right]\rangle
    \\ &=-\mathrm{i}\gamma^2\hbar H \langle [ J_z, \vec J ]\rangle
    \\ &=\gamma^2 \hbar H \langle J_y \vec e_x - J_x \vec e_y \rangle
    \\ &= \gamma H \langle \mu_x \vec e_y - \mu_y \vec e_x \rangle
    \\ &=\gamma \vec H \times \langle \vec \mu \rangle
    \end{align}
\item gleiche Bewegungsgleichung wie klassischer Dipol
\item Dipole wechselwirken nicht miteinander
\item isotropes System (kein Kristallfeld)
\item Gültigkeit der LS-Kopplung ($\vec J$)
\item auch gültig für Magnetisierung
\end{itemize}

\subsection{statisches Magnetfeld}

\begin{itemize}
\item statisches Magnetfeld $\vec H = H \vec e_z$
\item Bewegungsgleichungen
  \begin{align}
  \dot \mu_x &= -\gamma H \mu_y \\
  \dot \mu_y &= \gamma H \mu_x \\
  \dot \mu_z &= 0
  \end{align}
\item harmonischer Oszillator
\begin{align}
\ddot \mu_x &= -\gamma H \dot \mu_y = -(\gamma H)^2 \mu_x \\
\ddot \mu_y &= \gamma H \dot \mu_y = -(\gamma H)^2 \mu_y
\end{align}
\end{itemize}


\subsection{Lösung im rotierenden Magnetfeld}

\begin{itemize}
\item magnet. Resonanzexperimente benutzen zeitl. veränderliche Felder
\item experimentell üblich: Feld in $z$-Richtung konstant, $\cos{\omega t}$ in $x$-Richtung
\item Transformation
\begin{align}
\frac{\mathrm{d}\vec \mu}{\mathrm{d}t} &= \frac{\mathrm{d}\vec \mu'}{\mathrm{d}t} + \vec \omega \times \vec \mu \\
\frac{\mathrm{d}\vec \mu'}{\mathrm{d}t} &= \frac{\mathrm{d}\vec \mu}{\mathrm{d}t} - \vec \omega \times \vec \mu = \gamma \vec H \times \vec \mu - \vec \omega \times \vec \mu\\
 &= \gamma \underbrace{\left( \vec H - \frac{\vec \omega}{\gamma}\right)}_{\vec H_e} \times \vec \mu
\end{align}
\item $f = \frac{\gamma H_1}{2\pi}$, $\Rightarrow \frac{T}{2} = \frac{\pi}{\gamma H_1}$
\end{itemize}


\subsection{Bloch-Gleichungen}

\begin{itemize}
\item bei Resonanzexperimenten: Resonanz durchlaufen
\item eine Möglichkeit: $90^\mathrm{o}$-Puls
\item bis jetzt: unendliche Präzession
\item Präzession ist nicht kohärent über die ganze Probe
\item tatsächlich: Abnahme der Präzession
\item Einführen von Relaxationstermen
\item Annahme linearer Antwort (mathematisch einfach)
\item statistisches Problem $\Rightarrow$ Magnetisierung
\item Energie nur abhängig von $M_z$ $\Rightarrow$ Kopplung ans Gitter (Spin-Gitter-Relaxation)
\item $x$ und $y$ symmetrisch
\item Spin-Spin-Relaxations meistens schneller als Spin-Gitter-Relaxation
\end{itemize}
\end{document}
